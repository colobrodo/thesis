\chapter{Sistema di Testing}
    \section{Meccanismo dell'oracolo\label{oracolo}}
        Per la verifica del test viene sfruttato il meccanismo dell’oracolo.\\
        Esso consiste nell'accoppiare ogni test all'output corrispondente (chiamato appunto oracolo)
        e successivamente nell'effettuare un confronto tra il risultato del test e il suo riferimento.\\
        Il test viene dichiarato fallimentare se ci sono differenze tra gli outcome.\\
        
        Ogni test deve riferirsi solo a una funzionalit\'a o ad un insieme di esse strettamente correlate tra loro. \\
        Questo consente allo sviluppatore di individuare eventuali errori con maggiore precisione nel caso in cui un test fallisca. \\
        Se vi fosse la necessit\'a di cambiare le specifiche del software o di aggiungerne funzionalit\'a, \'e possibile rispettivamente modificare un oracolo esistente o crearne uno nuovo. \\

        Nel nostro sistema l'oracolo è salvato in un file \textit{.reference} e il risultato prodotto
        dal test in un file \textit{.current}.\\
        Il file current viene salvato nel file system solo nel caso in cui esso sia effettivamente diverso dal suo riferimento.\\
        
        Salvare il nuovo stato può consentire allo sviluppatore di confrontare i due file per capire quale sia l'errore introdotto nel software. \\
        Nella sezione successiva vengono illustrati gli elementi dal quale \'e composto un test e in seguito la loro funzione.\\
    \section{Anatomia di un test\label{testanatomy}}
        Un test consiste in un insieme di file locati per comodit\'a nella stessa cartella. \\

        Ogni test consiste in:
        \begin{itemize}
            \item Un file CAD di partenza
            \item Dei file per descrivere una configurazione specifica di un determinato cliente
            \item Un file SCL \footnote{\textbf{Sculptor Common Language} \'e un linguaggio di scripting proprietario interpretato con un insieme di routine per il disegno e la manifattura dei materiali} o un file python
            \item Uno o pi\'u riferimenti (file .reference)
        \end{itemize}

    \section{Ciclo di vita del test}
        Il software, per lanciare un test, importa il file CAD e le configurazioni del cliente.\\
        Successivamente esegue lo script SCL o Python, che con opportune librerie, si interfaccia con il software
        e simula le operazioni utente. \\
        
        Tra le diverse routine alcune si occupano di creare le reference esportando un risultato parziale dal programma come uno screenshot, un gruppo di entit\'a geometriche o un file testuale.

        Durante la esecuzione di un test i suoi outcome vengono salvati in una cartella di cache.\\
        Nel caso sia gi\'a presente una reference esso viene salvato come current.\\

        Se il file prodotto dalle operazioni \'e uguale all'oracolo, esso viene rimosso dal file system.\\
        In caso contrario il test \'e in stato di errore e l'output resta disponibile per il confronto.\\

    \section{Versioning dei tests}
            I test si trovano sotto un sistema di versioning.\\
            Quando uno sviluppatore decide di cambiare il comportamento di un caso d'uso o di aggiungere una funzionalità al software,
            sar\'a necessario modificare o creare un test.\\
            Esso verrà committato nel sistema di versioning in modo che gli altri sviluppatori possano testare il software secondo le nuove specifiche definite dal test.    
