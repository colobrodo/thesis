\chapter{Sistema di Testing}
    \section{Meccanismo dell'oracolo\label{oracolo}}
        Nel flusso di sviluppo attuale vi è una fase di test di integrazione.
        Per la verifica del test viene sfruttato il meccanismo dell’oracolo.
        Esso consiste nell'accoppiare ogni test all'output corrispondente (chiamato appunto oracolo)
        e successivamente nell'effettuare un confronto tra il risultato del test e il suo riferimento.
        Il test viene dichiarato fallimentare se ci sono differenze tra gli outcome.
        
        La particolarit\'a del meccanismo sta nella facolt\'a del test di riferirsi ad un unica funzionalit\'a o ad un insieme di esse strettamente correlate tra loro.
        Questo consente allo sviluppatore di individuare eventuali errori con maggiore precisione nel caso in cui il test fallisca. 
        Se vi fosse la necessit\'a di cambiare le specifiche del software o di aggiungerne funzionalit\'a, \'e possibile rispettivamente modificare un oracolo esistente o crearne uno nuovo. 

        Nel nostro sistema l'oracolo è salvato in un file \textit{.reference} e il risultato prodotto
        dal test in un file \textit{.current}.
        Il file current viene salvato nel file system solo nel caso in cui esso sia effettivamente diverso dal suo riferimento.
        
        Salvare il nuovo stato può consentire allo sviluppatore di confrontare i due file per capire quale sia l'errore introdotto nel software. \\
        Nella sezione successiva vengono illustrati gli elementi dal quale \'e composto un test e in seguito la loro funzione.
    \section{Anatomia di un test\label{testanatomy}}
        Un test consiste in un insieme di file locati per comodit\'a nella stessa cartella. 

        Ogni test consiste in:
        \begin{itemize}
            \item un file CAD di partenza
            \item dei file per descrivere una configurazione specifica di un determinato cliente
            \item un file scl (linguaggio di scripting proprietario) o un file python
            \item un riferimento (file .reference)
        \end{itemize}
    \section{Ciclo di vita del test}
        Il software, per lanciare un test, importa il file CAD e le configurazioni clienti.
        Successivamente esegue lo script SCL o Python, che con opportune librerie, si interfaccia con il software
        e simula le operazioni utente. 
        
        Una volta terminate le operazioni, il test salva il suo stato in un file e lo confronta con la sua reference 
        salvata nella cache del test. 

        Se il file prodotto dalle operazioni \'e uguale all'oracolo, esso viene rimosso dal file system.
        In caso contrario il test \'e in stato di errore e l'output resta disponibile per un confronto.

    \section{Versioning dei tests}
            I test si trovano sotto un sistema di versioning.
            Quando uno sviluppatore decide di cambiare il comportamento di un caso d'uso o di aggiungere una funzionalità al software,
            sar\'a necessario modificare o creare un test.\\
            Esso verrà committato nel sistema di versioning in modo che gli altri sviluppatori possano testare il software secondo le nuove specifiche definite dal test.    
            
    \section{Integrazione nel Workflow}
        \subsection{Fase Preliminare di sviluppo}
            I test sono suddivisi in caso d'uso.\\
            Pertanto, nella la maggior parte dei casi, lo sviluppatore riconosce i test che controlleranno la correttezza della sua modifica.\\
            In seguito verr\'a eseguito in locale solo un sottoinsieme di test al fine di ottenere una conferma preliminare della correttezza dello sviluppo. \\
        \subsection{Server di testing}
            Essendo la test suite molto ampia e il processo di testing molto dispendioso computazionalmente, 
            la sessione di test non viene lanciata in locale dal singolo sviluppatore, 
            Bens\'i  le modifiche del software già precedentemente compilate (sostituendo l’exe o una libreria dinamica .dll) 
            vengono caricate su un server apposito, il quale eseguir\'a ogni test della test suite.
            
            Nel caso in cui non ci siano test in stato di errore, la modifica al software viene committata e resa disponibile agli altri sviluppatori.
            
            Anche per questo motivo lo sviluppatore non interagisce direttamente sui singoli casi di test da file system,
            ma al fine di elencare i test fallimentari, di confrontarli e di committare eventuali cambiamenti negli oracoli, usa un report dinamico, 
            proprietario, oggetto di questa tesi.

            Il processo di esecuzione dei test sul server \'e delegato a uno script specifico.\\ 
            Questo si occupa di scaricare nuovamente i test da sistema di controllo di versione.\\
            Il programma sucessivamente lancia la suite su un'istanza del programma, per poi leggerne i file di log e notificare gli sviluppatori con una e-mail.\\
            Questa email conterr\'a delle informazioni preliminari, come il numero di test falliti, il tempo di esecuzione totale e un link al report dettagliato.
