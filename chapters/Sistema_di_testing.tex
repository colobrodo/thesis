\chapter{Sistema di Testing}
    \section{Meccanismo dell'oracolo\label{oracolo}}
        Nel flusso di sviluppo attuale vi è una fase di test di integrazione.
        Per la verifica del test viene sfruttato il meccanismo dell’oracolo.
        Esso consiste in accoppiare ogni test all'output aspettato (chiamato appunto oracolo),
        e successivamente effettuare un confronto tra il risultato del test e il suo riferimento.
        Il test viene dichiarato fallimentare se ci sono differenze tra questi outcome.
        
        L'idea è che ogni test rappresenti un caso d'uso, e il software deve mantenere un comportamento 
        consistente con la sua versione precedente.
        Nel caso questo comportamento voglia essere esplicitamente cambiato, viene modificato il test.

        Nel nostro sistema l'output aspettato è salvato in file \textit{.reference} e il risultato prodotto
        dal test in un file \textit{.current}.
        Il file current viene salvato nel file system solo se effettivamente diverso dal suo riferimento.
        
        Salvare il nuovo stato può consentire allo sviluppatore di confrontare i due file per capire quale sia l'errore introdotto nel software. \\
        Nella sezione successiva viene illustrato da quali elementi è composto un test e quale sia la loro funzione.
    \section{Anatomia di un test\label{testanatomy}}
        Un test consiste in un insieme di file locati per comodit\'a nella stessa cartella. 

        Ogni test consiste in:
        \begin{itemize}
            \item un file CAD di partenza
            \item dei file per specificare una configurazione specifica di un cliente
            \item un file scl (linguaggio di scripting proprietario) o un file python
            \item un riferimento (file .reference)
        \end{itemize}
    \section{Ciclo di vita del test}
        Il software, per lanciare un test, importa il file CAD e le configurazioni clienti.
        Successivamente esegue lo script SCL o Python, che con opportune librerie, si interfaccia al software
        e simula le operazioni utente. 
        
        Una volta terminate queste operazioni il test salva il suo stato in un file e lo confronta con la sua reference 
        salvata nella cache del test. 

        Se il file prodotto dalle operazioni \'e uguale all'oracolo aspettato, questo file viene rimosso dal file system, altrimenti il test \'e in errore e l'output viene lasciato disponibile per il confronto.

    \section{Versioning dei tests}
            I test si trovano sotto un sistema di versioning.
            Quando uno sviluppatore decide di cambiare il comportamento in un caso d'uso o aggiungere una funzionalità,
            modificher\'a o creer\'a rispettivamente un test.
            Esso verrà committato nel sistema di versioning in modo che gli altri sviluppatori possano testare il software secondo le nuove specifiche definite dal test.    
            
    \section{Integrazione nel Workflow}
        \subsection{Fase Preliminare di sviluppo}
            I test sono suddivisi in caso d'uso quindi, per la maggior parte dei casi, lo sviluppatore sa quali test controlleranno la correttezza della sua modifica.
            Lo sviluppatore eseguirà in locale quindi solo un sottoinsieme di test per avere una conferma preliminare della correttezza dello sviluppo. 
        \subsection{Server di testing}
            Essendo la test suite molto grossa e il processo di testing molto dispendioso, 
            la sessione di test viene lanciata, non in locale dal singolo sviluppatore, 
            ma le modifiche del software già compilate (sostituendo l’exe o una libreria dinamica .dll) 
            vengono caricate su un server dedicato che eseguirà ogni test della test suite.
            
            Se i test passano la modifica al software viene committata, e resa disponibile agli altri sviluppatori.
            
            Anche per questo motivo lo sviluppatore non interagisce sui singoli casi di test da file system,
            ma per elencare i test fallimentari, confrontarli e committare eventuali cambiamenti nelle references, usa un report dinamico, 
            proprietario, oggetto di questa tesi.

            Il processo di esecuzione dei test sul server \'e delegato a uno script dedicato che si occupa di ottenere lo stato corrente dei test da controllo di versione,
            eseguire la suite su un istanza del programma, in seguito leggerne i log e notificare gli sviluppatori con una e-mail.
            Questa email conterr\'a delle informazioni preliminari, come il numero di test falliti e il tempo di esecuzione totale, e un link al report dettagliato.
