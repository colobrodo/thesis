\chapter{Sviluppi Futuri}
    \section{Migrazione test suite da svn a git}
        Le test suite sono attualmente revisionate dal sistema SVN che dispone di un controllo di versione molto piu elementare rispetto a GIT.\\

        Subversion \'e stato precedentemente utilizzato per la revisione di tutti i progetti all'interno dell'azienda, test inclusi.\\
        Attualmente si sta lentamente migrando verso l'uso estensivo di GIT, che offre molte pi\'u funzionalit\'a.\\

        Il test report favorisce questa migrazione utilizzando in modo trasparente all'utente entrambi i sistemi di versione. 
    \section{Performance dei test}
        Il software ha la possibililtà di tenere traccia del tempo di esecuzione di un singolo test.\\
        Attualmente le performance di ogni test viene scritta in un database flat-file \footnote{Database con una struttura uniforme che vengono salvati in un unico file binario}.

        Se un tempo supera una soglia che viene considerata il tempo di esecuzione standard di un test questo viene segnalato nella mail di notifica.\\
        L'intenzione futura sarebbe includere nel report i tempi di esecuzione dei singoli test e segnalare quelli con un tempo di esecuzione troppo alto.\\
    \section{Differ integrato}
        Una proposta per migliorare il report di test \'e quella di reimplementare il sistema di differenze senza riusare quello già esistente.\\
        Questo rimuoverebbe il problema di allineare i due sistemi passando il token di sessione a entrambi.\\
        Inolte in questo modo il report di test non puo garantire di essere sincronizzato con il differ esterno: se vengono effettuate modifiche (anche da controllo di versione) sui file reference il sistema non puo visualizzarne le differenze in tempo reale.\\
        
        Un differ integrato si assumerebbe l'onere di calcolare le differenze tra i due file binari ed interpretarli correttamente in modo da esporre nell'API un confronto tra i file leggibile.\\
        Questo consentirebbe di poter confrontare i file in qualsiasi momento senza caricarne prima le differenze su un server apposito.
    \section{Raggruppamento per cartelle}
        Alcune test suite si occupano di testare diversi moduli all'interno dell'applicativo.\\
        I test adibiti a un singolo modulo vengono raggruppati nella stessa cartella.\\
        Un membro di un team di sviluppo si occupa del mantenimento di un insieme ristretto di moduli.\\
        Dunque una necessit\'a molto diffusa \'e visualizzare solo i test di un modulo oppure escludere dalla visualizzazione i test di moduli non interessati. 
        Per soddisfare questa specifica si potrebbe disporre un layout di raggruppamento dei test per cartella.\\
        Questo consentirebbe di poter visualizzare e gestire i test in base al loro modulo di appartenenza.\\ 

        La complicanza tecnica di questa suddivisione consiste nel separare le modifiche attualmente registrate nel sistema di versioning per il precommit.\\
        Il commit delle modifiche di un modulo non deve registrare anche operazioni effettuate su test in altre cartelle ma tutta la suite \'e versionata nella stessa repository.\\
        Questo implica la creazione di multiple aree di stage che devono essere implementate ad-hoc, essendo questa una funzionalità non presente in nessun sistema di controllo di versione.\\