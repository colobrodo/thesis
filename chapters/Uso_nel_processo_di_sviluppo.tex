\chapter{Distribuzione e uso nel processo di sviluppo}

    \section{Bundling e deploy}
        Ogni team che si occupa dello sviluppo di un'applicazione di CAD/CAM ha adibito un server interno per la compilazione e il testing del proprio programma.\\
        L'applicativo di report dei test è stato installato su ogni server.\\

        Essendo l'applicazione web-based è stata adibita una porta dove servire la pagina.\\
        Sulla stessa porta è stata esposta anche l'API.\\

        In fase di sviluppo del report vengono usati due processi, uno per il frontend e uno per il backend, che comunicano tra loro tramite messaggi HTTP.\\
        Questa gestione è diversa in fase di produzione siccome il processo che serve la parte frontend si occupa di ricaricare la pagina quando occorre una modifica sui file del progetto angular, in modo da mostrare le modifiche in modo interattivo allo sviluppatore.\\
        Per l'interfaccia web viene dunque effettuata una fase detta di "bundling", dove i diversi file  vengono compilati in file statici (JavaScript \cite{JS}, HTML \cite{HTML} e CSS \cite{CSS}).\\
        Vengono esposte entrambe le parti dell'applicazione sulla stessa porta grazie al sistema di routing.\\
        Tutti gli indirizzi relativi all'API sono stati spostati nella sottoroute \verb|/v1|.\\
        Per le altre richieste viene controllato se esiste un file statico con lo stesso percorso altrimenti viene servito il file HTML principale in modo da delegare la segnalazione di errori o visualizzazione di pagine specifiche al sistema di routing usata dalla Single Page Application.\\

    \section{Processo di esecuzione dei test\label{testexecution}}
        I test vengono lanciati usando uno script.\\
        Esso si occupa di lanciare il programma ed eseguire tutti i test in ogni suite.\\
        Successivamente carica i risultati su un server di storage apposito e notifica lo sviluppatore con una mail al termine della sessione.\\

        La mail di notifica contiene informazioni preliminari quali il numero di test in errore e un link alla pagine del test report corrispondente.\\

        L'integrazione tra il differ e il report viene gestita da questo programma.\\
        Durante la fase di upload viene creato un token per la sessione appena terminata.\\
        Per ogni test in errore caricato sul server è creata una pagina HTML con le differenze tra le reference e i current.\\
        
        L'indirizzo di ogni pagina è composto dal token di sessione e il nome del test.\\
        Nel link relativo al test report viene passato come query param il token della sessione e il report creer\'a l'indirizzo relativo alla pagina delle differenze per ogni elemento in stato di errore.\\
