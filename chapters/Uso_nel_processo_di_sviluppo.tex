\chapter{Distribuzione e uso nel processo di sviluppo}
    \section{Flusso di sviluppo}
        Per eseguire una modifica al software e renderla pubblica gli sviluppatori seguono un flusso di sviluppo prestabilito.
        L'ultima versione pubblica viene scaricata in locale usando il sistema di controllo di versione.
        Lo sviluppatore carica i moduli necessari all'interno dell'IDE.
        Ora vengono scritte o modificate le porzioni di codice per la correzzione dell'errore o l'aggiunta della funzionalit\'a.
        Dal sorgente modificato viene compilata una versione di debug del software che \'e quella verranno testate in modo preliminare le modifiche.
        Al termine della fase di scrittura del codice, per testare l'effettiva correttezza in ogni caso d'uso viene compilato il modulo modificato in una versione \textit{release} sia a 32 che 64 bit.
        Il file compilato, che sia eseguibile o \textit{DLL} viene sostituito sul server di test.
        Ora lo sviluppatore lancia tutti i test presenti nella suite come descritto pi\'u dettagliatamente nel capitolo \ref{testexecution}.
        Verrà notificato con un mail contentente un link al report e se i test hanno prodotto il risultato sperato, le modifiche verranno revisionate nel sistema di controllo di versione.
        Altrimenti sar\'a necessario debuggare i test con esito negativo per capire quali sono i comportamenti introdotti che non seguono le specifiche. 

    \section{Bundling e deploy}
        Ogni team che si occupa dello sviluppo di un'applicazione di CAD/CAM ha adibito un server interno per la compilazione e il testing del proprio programma.
        L'applicativo di report dei test è stato installato su ogni server.
        essendo l'applicazione web-based è stata adibita una porta dove servire la pagina.
        sulla stessa porta è stata esposta anche l'API.
        In fase di sviluppo vengono usati due processi, uno per il frontend e uno per il backend, che comunicano tra loro tramite messaggi HTTP.
        Questa gestione è diversa in fase di produzione siccome il processo che serve la parte frontend si occupa di ricaricare la pagina quando occorre una modifica sui file del progetto angular, in modo da mostrare le modifiche in modo interattivo allo sviluppatore.
        Per l'interfaccia web viene dunque effettuata una fase detta di "bundling", dove i diversi file  vengono compilati in file statici (js HTML e CSS).
        Vengono esposte entrambe le parti dell'applicazione sulla stessa porta grazie al sistema di routing.
        Tutti gli indirizzi relativi all'API sono stati spostati nella sottoroute \verb|/v1|
        Per le altre richieste viene controllato se esiste un file statico con lo stesso percorso altrimenti viene servito il file HTML principale in modo da delegare la segnalazione di errori o visualizzazione di pagine specifiche al sistema di routing usata dalla Single Page Application.

    \section{Processo di esecuzione dei test\label{testexecution}}
        I test vengono lanciati usando uno script.
        Questo script si occupa di lanciare il programma e eseguire tutti i test in ogni suite.
        Successivamente carica i risultati su un server di storage apposito e notifica lo sviluppatore con una mail al termine della sessione.
        La mail di notifica contiene informazioni preliminari quali il numero di test in errore e un link alla pagine del test report corrispondente.
        L'integrazione tra il differ e il report viene gestita da questo programma.
        Durante il caricamento viene creato un token per la sessione appena terminata.
        Per ogni test in errore caricato sul server è creata dinamica una pagina HTML con le differenze e l'indirizzo di questa pagina è composto dal token di sessione e il nome del test.
        Nel link relativo al test report viene passato come query param il token della sessione e il report comporra l'indirizzo relativo alla pagina delle differenze per ogni elemento in stato di errore.
