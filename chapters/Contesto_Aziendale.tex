\chapter{Contesto aziendale di sviluppo}
    Ogni programma sviluppato internamente si occupa in modo specializzato della manifattura di un materiale specifico (come il legno, la pietra o il vetro).\\
    Lo sviluppo dei software CAD/CAM \'e stato suddiviso in diversi moduli.\\
    Alcuni per esempio si occupano delle funzionalit\'a comuni di disegno e di manipolazione delle geometrie, oppure del calcolo della posizione degli utensili durante la lavorazione del materiale.\\
    Questo tipo di moduli che provvedono funzionalit\'a comuni sono utilizzati da multipli software.\\
    Altri moduli (detti plugin) sono opzionali e usano le funzionalità del software per automatizzare processi industriali come, nel caso del legno, la produzione di travi o di mobili da cucina.\\
    
    Nella fase di testing ogni modulo \'e testato separatamente in una test suite a lui dedicata.\\

    \section{Flusso di sviluppo}
        Per eseguire una modifica al software e renderla pubblica gli sviluppatori seguono un flusso di sviluppo prestabilito.\\
        L'ultima versione pubblica viene scaricata in locale usando il sistema di controllo di versione.\\
        Lo sviluppatore carica i moduli necessari all'interno dell'IDE.\\
        Ora vengono scritte o modificate le porzioni di codice necessarie per la correzzione dell'errore o l'aggiunta della funzionalit\'a.\\
        Dal sorgente modificato viene compilata una versione di debug del software che verr\'a usata per lo sviluppo.\\
        Al termine della fase di scrittura del codice, per testare l'effettiva correttezza in ogni caso d'uso viene compilato il modulo modificato in una versione \textit{release} sia a 32 che 64 bit.
        La versione di release differisce da quella di debug perch\'e non dispone di tutte le operazioni di log e telemetrica garantendo delle maggiori performance del software.\\
        
        I test sono suddivisi in caso d'uso.\\
        Pertanto, nella la maggior parte dei casi, lo sviluppatore sa quali test  controlleranno la correttezza della sua modifica.\\
        Generalmente viene eseguito in locale solo un sottoinsieme di test al fine di ottenere una conferma preliminare della correttezza dello sviluppo.\\

        Essendo la test suite molto ampia e il processo di testing molto dispendioso computazionalmente, 
        la sessione di test non viene lanciata in locale dal singolo sviluppatore, 
        Bens\'i  le modifiche del software già precedentemente compilate (sostituendo l’exe o una libreria dinamica .dll) 
        vengono caricate su un server apposito, il quale eseguir\'a ogni test della test suite.\\

        Il processo di testing su server \'e descritto pi\'u dettagliatamente nel capitolo \ref{testexecution}.\\
        
        Al termine dell'esecuzione dei test gli sviluppatori interessati vengono notificati con un mail contentente un link al report.\\
        Attraverso l'uso del report il programmatore decide se i test hanno prodotto il risultato sperato.\\
        Nel miglior caso le modifiche verranno revisionate nel sistema di controllo di versione.\\
        Altrimenti e\' necessario debuggare i test con esito negativo per capire quali sono i comportamenti introdotti che non seguono le specifiche. \\

    \section{Necessit\'a alla base dello sviluppo}
        Prima dello sviluppo era gi\'a presente un tool web a supporto della fase di analisi e accettazione dei test.\\
        Questo strumento aveva per\'o le seguenti limitazioni:\\
        \begin{itemize}
            \item Il sistema di versione GIT non è supportato.
            \item il tool è svilupparto con tecnologie di rendering server-side e ogni operazione richiede il ricaricamento della pagina e di conseguenza l'intera scansione della test suite.
            \item Il rendering server side non rendeva riusabili le informazioni riguardo hai test.
            \item L'interfaccia grafica non \'e interattiva e non è possibile nascondere o filtrare i test.
        \end{itemize}

        Per questi motivi \'e stato scelto di sviluppare un applicativo con un architettura client-server (Descritta dettagliatamente nel capitolo \ref{appstruct}).