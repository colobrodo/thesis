\chapter{Struttura dell'applicazione}
    L'applicazione \'e stata sviluppata seguendo un'architettura client-server.\\
    
    Questa infrastruttura è composta da diversi programmi che interagiscono tra loro usando messaggi di rete, nel nostro caso codificati secondo il protocollo HTTP.
    Comunemente, le componenti principali sono il backend e il frontend.\\

    \section{Elementi dell'applicazione}
        La prima componente di \textit{backend} si occupa della parte di logica di dominio dell'applicazione.\\
        La seconda, denominata \textit{frontend}, rappresenta graficamente i dati e di rende le interazioni tra essi gli stessi e l'utente fruibili.
        Decuplicare le parti dell'applicazione consente di rendere i singoli componenti riusabili.\\

        L'esempio pi\'u comune consiste nel condividere la logica dell'applicazione usando un backend e multiple rappresentazioni grafiche a seconda della piattaforma di utilizzo.
        In questo modo si può creare a un'applicazione web o un app mobile, senza duplicare la logica di dominio che verrà gestita dell'altra componente.
        
        Nel nostro caso in esame, la parte di backend si occuperà di leggere i test da file system e di esporli tramite un API.\\
        \'E stato sviluppato un frontend web per interagire graficamente con i test.\\

    \section{Intercomunicazione tra gli elementi}
        La comunicazione avviene tramite messaggi di rete codificati secondo il protocollo HTTP.\\
        La condivisione del protocollo consente agli attori di comunicare senza conoscere i dettagli implementativi dell'altro.\\
        Ci\'o implica un ulteriore vantaggio: gli elementi possono essere eseguiti su piattaforme diverse o essere scritti in linguaggi differenti, a seconda delle diverse necessità del singolo componente.\\
        
        Per il caso in oggetto la comunicazione viene iniziata dall'applicazione web e l'API REST si occupa di rispondere alle richieste.\\
        Il formato di interscambio dei dati usato \'e \textit{JSON}\cite{JSON}.

    \section{Vantaggi, svantaggi dell'architettura e alternative}
        I vantaggi di questa architettura, come precedentemente detto, sono la decuplicazione dei diversi elementi, il loro riutilizzo e la separazione delle responsabilità.\\
        
        In un caso futuro, sarà possibile sviluppare un altro frontend per visualizzare i test su un altra piattaforma riutilizzando il backend e garantendo la stessa logica applicativa.\\
        Viceversa la parte di frontend non conosce i dettagli implementativi usati dal backend come la gestione delle reference tramite file system e il sistema di versioning.\\
        Questo consente di cambiare le gestioni di basso livello lasciando invariato l'uso del programma agli occhi dell'utente.\\  
        
        Dall'altra parte, di contro, la separazione degli elementi porta a una piccola duplicazione di quelle che sono le parti comuni tra il client e il server, come i modelli di dominio.\\

        Prima dello sviluppo era gi\'a presente un tool web a supporto della fase di analisi e accettazione dei test.\\
        Questo strumento aveva per\'o le seguenti limitazioni:\\
            \begin{itemize}
                \item Il sistema di versione GIT non è supportato.
                \item il tool è svilupparto con tecnologie di rendering server-side e ogni operazione richiede il ricaricamento della pagina e di conseguenza l'intera scansione della test suite.
                \item Il rendering server side non rendeva disponibili esternamente le informazioni riguardo hai test.
                \item L'interfaccia grafica non \'e interattiva e non è possibile nascondere o filtrare i test.
            \end{itemize}

        Per il tool aziendale precendente era stata adottata un'architettura diversa: un unico attore si occupava di consultare le reference e creare l'interfaccia grafica. \\
        Questo sistema non \'e per\'o estensibile e riusabile come quello attuale, essendo la logica di dominio e la generazione della pagina eseguita dallo stesso componente.
