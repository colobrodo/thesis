\chapter{Struttura dell'applicazione}
    L'applicazione segue una architettura client-server.
    Questa infrastruttura è composta da diversi programmi che interagiscono tra loro usando messaggi di rete.
    Comunemente le componenti principali sono quelle di backend e di frontend.    

\section{Elementi dell'applicazione}
    La prima di occupano della parte di logica di dominio dell'applicazione.
    La seconda di rappresentare graficamente i dati e rendere le interazioni tra essi e l'utente fruibili.
    Decuplicare le parti dell'applicazione consente di rendere i singoli componenti riusabili, 
    l'esempio pi\'u comune \'e due client, uno web e uno mobile, che si interfacciano allo stesso backend.
    Nel nostro caso in esame la parte di backend si occuperà di leggere i test da file system e esporli trammite un API 
    che verrà consultata da una SPA \footnote{una single-page application (SPA) \'e una applicazione web che interagisce con l'utente dinamicamente ridisegnando la pagina corrente creando la struttura della pagina da un linguaggio di scripting e non delegando il caricamento al browser il caricamento del file HTML} web-based.

\section{Intercomunicazione tra gli elementi}
    La comunicazione avviene tramite protocollo HTTP, ovvero messaggi di rete.
    La condivisione del protocollo consente agli attori di comunicare senza conoscere i dettagli implementativi dell'altro.
    Ci\'o porta a un altro vantaggio: gli elementi possono risiedere su strutture diverse o essere scritti in linguaggi differenti a seconda delle diverse necessità del programma.
    Le parti principali di un messaggio HTTP sono:
    \begin{itemize}
        \item \textbf{l'indirizzo:} che indica la risorsa a cui fa riferimento la richiesta
        \item \textbf{Il metodo:} ovvero il tipo di azione da eseguire 
        \item \textbf{Il corpo:} Dati di parametro per una richiesta o di risultato per una risposta. 
    \end{itemize}

\section{Vantaggi, svantaggi dell'architettura e alternative}
    I vantaggi di questa architettura, come precedentemente detto, sono la decuplicazione dei diversi elementi, e il loro riuso e la separazione delle responsabilità.
    Nel caso in futuro uno sviluppatore voglia sviluppare un plugin per ottenere lo stato dei test direttamente da IDE, lo potrà fare riusando le API prodotte.
    Dall'altra parte, di contro, la separazione degli elementi porta una piccola duplicazione in quelle che sono le parti comuni tra il client e il server, come i modelli di dominio.
    Un'alternativa precedentemente usata per un tool aziendale e l'implementazione di una interfaccia web generata in modo statico da un server.
    In questo modo in base hai parametri il  server crerà un documento HTML specifico.
    Questo sistema non \'e per\'o estensibile e riusabile come REST.
