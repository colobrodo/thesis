\chapter{Frontend}
    \section{Angular}
        \subsection{Scelta dell'utilizzo}
            \'E stato scelto di usare Angular, perchè è un framework stabile supportato da Google
            e \'e già stato usato precedentemente per lo sviluppo di altri applicativi.
            
            La struttura a componenti, favorisce il riuso del codice perciò è stato possibile integrare
            nello sviluppo componenti già sviluppati da altri, che forniscono comportamenti base
            come il rendering 3d di un file CAD proprietario.
        
        \subsection{Typescript}
            Un altro motivo per la scelta di Angular è \href{https://www.typescriptlang.org}{Typescript}.
            Typescript è un \textit{superset di javascript} cioè una sua estensione che ingloba tutte le sue feature e ne aggiunge altre.
            La sua peculiarità principale è l'aggiunta di uno step di compilazione che consente di aggiungere una tipizzazione statica.
            Questa tipizzazione consente allo sviluppatore di
            \href{https://www.quora.com/Why-is-type-checking-important-in-programming-languages-and-how-should-one-choose-between-dynamically-and-statically-typed-languages}{ridurre errori banali}.
    \section{Funzionalità}
        \subsection{visualizzazione dei test}         
        \subsection{Gestione interattiva dei file stageati}         
        \subsection{Modalità compatta}
        \subsection{Navigazione}
    \section{Interazione con gli altri tool aziendali}