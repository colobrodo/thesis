\chapter{Frontend}
    \section{Angular}
        \subsection{Scelta dell'utilizzo}
            \'E stato scelto di usare Angular, perchè è un framework stabile supportato da Google
            e \'e già stato usato precedentemente per lo sviluppo di altri applicativi.
            
            La struttura a componenti, favorisce il riuso del codice perciò è stato possibile integrare
            nello sviluppo componenti già sviluppati da altri, che forniscono comportamenti base
            come il rendering 3d di un file CAD proprietario.
        
        \subsection{Typescript}
            Un altro motivo per la scelta di Angular è \href{https://www.typescriptlang.org}{Typescript}.
            Typescript è un \textit{superset di javascript} cioè una sua estensione che ingloba tutte le sue feature e ne aggiunge altre.
            La sua peculiarità principale è l'aggiunta di uno step di compilazione che consente di aggiungere una tipizzazione statica.
            Questa tipizzazione consente allo sviluppatore di
            \href{https://www.quora.com/Why-is-type-checking-important-in-programming-languages-and-how-should-one-choose-between-dynamically-and-statically-typed-languages}{ridurre errori banali}.
    \section{Funzionalità}
        \subsection{visualizzazione dei test}         
        \subsection{Gestione interattiva dei file stageati}         
        \subsection{Modalità compatta}
        \subsection{Navigazione}
            E\'  disponibile un metodo di navigazione per aumentare l'usabilità dell'applicazione.
            Durante la navigazione è possibile rendere un test \textit{focussed} e effettuare operazioni (come nasconderlo o metterlo in stage) usando delle scorciatoie da tastiera.\\
            Si può spostare il focus sul test precedente e successivo sempre usando shortcuts.
            Solo un test può essere in focus.\\
            Quando il focus si sposta su un nuovo test la pagina effettua un operazione di scroll per renderlo visibile e scrive il suo nome nell'url, nella sezione denominata \textit{fragment}.\\
            Se la pagina viene richiesta con un fragment già impostato, in fase di rendering dei tests viene focussato automaticamente il test con quel nome (e quindi incluso nella sezione visibile della pagina).\\
            Questo consente agli utenti di condividere un test trammite url.

            TODO: immagine test navigazione e url con fragment
    \section{Interazione con gli altri tool aziendali}